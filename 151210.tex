%%%%%%%%%%%%%%%%%%%%%%%%%%%%%%%%%%%%%%%%%%%%%%%%%%%%%%%%%%%%%%%%%%%%%%%%
%%%%%%%%%%%%%%%%%%%%%% Simple LaTeX CV Template %%%%%%%%%%%%%%%%%%%%%%%%
%%%%%%%%%%%%%%%%%%%%%%%%%%%%%%%%%%%%%%%%%%%%%%%%%%%%%%%%%%%%%%%%%%%%%%%%

%%%%%%%%%%%%%%%%%%%%%%%%%%%%%%%%%%%%%%%%%%%%%%%%%%%%%%%%%%%%%%%%%%%%%%%%
%% NOTE: If you find that it says                                     %%
%%                                                                    %%
%%                           1 of ??                                  %%
%%                                                                    %%
%% at the bottom of your first page, this means that the AUX file     %%
%% was not available when you ran LaTeX on this source. Simply RERUN  %%
%% LaTeX to get the ``??'' replaced with the number of the last page  %%
%% of the document. The AUX file will be generated on the first run   %%
%% of LaTeX and used on the second run to fill in all of the          %%
%% references.                                                        %%
%%%%%%%%%%%%%%%%%%%%%%%%%%%%%%%%%%%%%%%%%%%%%%%%%%%%%%%%%%%%%%%%%%%%%%%%

%%%%%%%%%%%%%%%%%%%%%%%%%%%% Document Setup %%%%%%%%%%%%%%%%%%%%%%%%%%%%

% Don't like 10pt? Try 11pt or 12pt
\documentclass[10pt]{article}

% The automated optical recognition software used to digitize resume
% information works best with fonts that do not have serifs. This
% command uses a sans serif font throughout. Uncomment both lines (or at
% least the second) to restore a Roman font (i.e., a font with serifs).
%\usepackage{times}
%\renewcommand{\familydefault}{\sfdefault}
%\usepackage{kotex}
%\usepackage{fontspec}
%\defaultfontfeatures{Mapping=tex-text}
%\setmainfont{Minion Pro}

% This is a helpful package that puts math inside length specifications
\usepackage{calc}
\usepackage{comment}

% This is a helpful package that puts graphical objects inside the document
\usepackage{graphicx}
\graphicspath{{.}{.}}
\DeclareGraphicsExtensions{.pdf,.jpg,.png}

% Simpler bibsection for CV sections
% (thanks to natbib for inspiration)
\linespread{0.97}
\makeatletter
\newlength{\bibhang}	
\setlength{\bibhang}{1em} %1em}
\newlength{\bibsep}
 {\@listi \global\bibsep\itemsep \global\advance\bibsep by\parsep}
\newenvironment{bibsection}%
        {\begin{enumerate}{}{%
%        {\begin{list}{}{%
       \setlength{\leftmargin}{\bibhang}%
       \setlength{\itemindent}{-\leftmargin}%
       \setlength{\itemsep}{\bibsep}%
       \setlength{\parsep}{\z@}%
        \setlength{\partopsep}{0pt}%
        \setlength{\topsep}{0pt}}}
        {\end{enumerate}\vspace{-.6\baselineskip}}
%        {\end{list}\vspace{-.6\baselineskip}}
\makeatother

% Layout: Puts the section titles on left side of page
\reversemarginpar
\usepackage{array}
\usepackage{lipsum, tikz}
\usetikzlibrary{arrows}
%\usepackage[object=vect%orian]{pgfornament}
%
%         PAPER SIZE, PAGE NUMBER, AND DOCUMENT LAYOUT NOTES:
%
% The next \usepackage line changes the layout for CV style section
% headings as marginal notes. It also sets up the paper size as either
% letter or A4. By default, letter was used. If A4 paper is desired,
% comment out the letterpaper lines and uncomment the a4paper lines.
%
% As you can see, the margin widths and section title widths can be
% easily adjusted.
%
% ALSO: Notice that the includefoot option can be commented OUT in order
% to put the PAGE NUMBER *IN* the bottom margin. This will make the
% effective text area larger.
%
% IF YOU WISH TO REMOVE THE ``of LASTPAGE'' next to each page number,
% see the note about the +LP and -LP lines below. Comment out the +LP
% and uncomment the -LP.
%
% IF YOU WISH TO REMOVE PAGE NUMBERS, be sure that the includefoot line
% is uncommented and ALSO uncomment the \pagestyle{empty} a few lines
% below.
%

%% Use these lines for letter-sized paper
\usepackage[paper=letterpaper,
            %includefoot, % Uncomment to put page number above margin
            marginparwidth=1.05in,     % Length of section titles
            marginparsep=0.06in,       % Space between titles and text
            margin=0.7in,               % 1 inch margins
            voffset=0.1in,
            includemp]{geometry}

%% Use these lines for A4-sized paper
%\usepackage[paper=a4paper,
%            %includefoot, % Uncomment to put page number above margin
%            marginparwidth=30.5mm,    % Length of section titles
%            marginparsep=1.5mm,       % Space between titles and text
%            margin=25mm,              % 25mm margins
%            includemp]{geometry}

%% More layout: Get rid of indenting throughout entire document
\setlength{\parindent}{0in}

\usepackage[shortlabels]{enumitem}

%% Reference the last page in the page number
%
% NOTE: comment the +LP line and uncomment the -LP line to have page
%       numbers without the ``of ##'' last page reference)
%
% NOTE: uncomment the \pagestyle{empty} line to get rid of all page
%       numbers (make sure includefoot is commented out above)
%
\usepackage{fancyhdr,lastpage}
\pagestyle{fancy}
\pagestyle{empty}      % Uncomment this to get rid of page numbers
\fancyhf{}\renewcommand{\headrulewidth}{0pt}
\fancyfootoffset{\marginparsep+\marginparwidth}
\newlength{\footpageshift}
\setlength{\footpageshift}
          {0.5\textwidth+0.5\marginparsep+0.5\marginparwidth-2in}
\lfoot{\hspace{\footpageshift}%
       \parbox{4in}{\, \hfill %
                    \arabic{page} of \protect\pageref*{LastPage} % +LP
%                    \arabic{page}                               % -LP
                    \hfill \,}}

% Finally, give us PDF bookmarks
\usepackage{color,hyperref}
\definecolor{darkblue}{rgb}{0.0,0.0,0.3}
\hypersetup{colorlinks,breaklinks,
            linkcolor=darkblue,urlcolor=darkblue,
            anchorcolor=darkblue,citecolor=darkblue}

%%%%%%%%%%%%%%%%%%%%%%%% End Document Setup %%%%%%%%%%%%%%%%%%%%%%%%%%%%


%%%%%%%%%%%%%%%%%%%%%%%%%%% Helper Commands %%%%%%%%%%%%%%%%%%%%%%%%%%%%

% The title (name) with a horizontal rule under it
% (optional argument typesets an object right-justified across from name
%  as well)
%
% Usage: \makeheading{name}
%        OR
%        \makeheading[right_object]{name}
%
% Place at top of document. It should be the first thing.
% If ``right_object'' is provided in the square-braced optional
% argument, it will be right justified on the same line as ``name'' at
% the top of the CV. For example:
%
%       \makeheading[\emph{Curriculum vitae}]{Your Name}
%
% will put an emphasized ``Curriculum vitae'' at the top of the document
% as a title. Likewise, a picture could be included:
%
%   \makeheading[\includegraphics[height=1.5in]{my_picutre}]{Your Name}
%
% the picture will be flush right across from the name.
\newcommand{\makeheading}[2]%
        {\hspace*{-\marginparsep minus \marginparwidth}%
        \begin{minipage}[t]{\textwidth+\marginparwidth+\marginparsep}%
        \centering{\LARGE{#2}}%
        \end{minipage}}
        

% The section headings
%
% Usage: \section{section name}
\renewcommand{\section}[1]{\pagebreak[3]%
    \hyphenpenalty=10000%
    \vspace{1\baselineskip}%
    \phantomsection\addcontentsline{toc}{section}{#1}%
    \noindent\llap{\scshape\smash{\parbox[t]{\marginparwidth}{\raggedright #1}}}%
    \vspace{-\baselineskip}\par}

% An itemize-style list with lots of space between items
\newenvironment{outerlist}[1][\enskip\textbullet]%
        {\begin{itemize}[#1,leftmargin=*,parsep=1pt,itemsep=1pt,topsep=1pt,partopsep=1pt]}{\end{itemize}}

% An environment IDENTICAL to outerlist that has better pre-list spacing
% when used as the first thing in a \section
\newenvironment{lonelist}[1][\enskip\textbullet]%
        {\begin{list}{#1}{%
        \setlength{\partopsep}{0pt}%
        \setlength{\topsep}{0pt}}}
        {\end{list}\vspace{-.5baselineskip}}
        
%,itemsep=0pt,topsep=0pt
% An itemize-style list with little space between items
\newenvironment{innerlist}[1][\enskip\textbullet]%
        {\begin{itemize}[#1,leftmargin=*,parsep=1pt,itemsep=1pt,topsep=1pt,partopsep=1pt]}
        {\end{itemize}}
%        {\vspace{-.5\baselineskip}
%\vspace{-.5\baselineskip}
% An environment IDENTICAL to innerlist that has better pre-list spacing
% when used as the first thing in a \section
\newenvironment{loneinnerlist}[1][\enskip\textbullet]%
        {\begin{itemize}[#1,leftmargin=*,parsep=1pt,itemsep=1pt,topsep=1pt,partopsep=1pt]}
        {\end{itemize}\vspace{-.6\baselineskip}}
        
\newcommand{\myrule} [3] []{
    \begin{center}
        \begin{tikzpicture}
            \draw[#2-#3, ultra thick, #1] (0,0) to (\linewidth,0);
        \end{tikzpicture}
    \end{center}
}

\newcommand{\sectionline}[2]{%
  \nointerlineskip \vspace{.5\baselineskip}\hspace{\fill}
  {\color{#1}
    \resizebox{\linewidth}{2ex}
    {{%
    {\begin{tikzpicture}
    \node  (C) at (0,0) {};
    \node (D) at (9,0) {};
    \path (C) to [ornament=#2] (D);
    \end{tikzpicture}}}}}%
    \hspace{\fill}
    \par\nointerlineskip \vspace{-\baselineskip}
}

% To add some paragraph space between lines.
% This also tells LaTeX to preferably break a page on one of these gaps
% if there is a needed pagebreak nearby.
\newcommand{\blankline}{\quad\pagebreak[3]}
\newcommand{\halfblankline}{\quad\vspace{-0.5\baselineskip}\pagebreak[3]}

% Uses hyperref to link DOI
\newcommand\doilink[1]{\href{http://dx.doi.org/#1}{#1}}
\newcommand\doi[1]{doi:\doilink{#1}}

% For \url{SOME_URL}, links SOME_URL to the url SOME_URL
\providecommand*\url[1]{\href{#1}{#1}}
% Same as above, but pretty-prints SOME_URL in teletype fixed-width font
\renewcommand*\url[1]{\href{#1}{\texttt{#1}}}

% For \email{ADDRESS}, links ADDRESS to the url mailto:ADDRESS
\providecommand*\email[1]{\href{mailto:#1}{#1}}
% Same as above, but pretty-prints ADDRESS in teletype fixed-width font
%\renewcommand*\email[1]{\href{mailto:#1}{\texttt{#1}}}

%\providecommand\BibTeX{{\rm B\kern-.05em{\sc i\kern-.025em b}\kern-.08em
%    T\kern-.1667em\lower.7ex\hbox{E}\kern-.125emX}}
%\providecommand\BibTeX{{\rm B\kern-.05em{\sc i\kern-.025em b}\kern-.08em
%    \TeX}}
\providecommand\BibTeX{{B\kern-.05em{\sc i\kern-.025em b}\kern-.08em
    \TeX}}
\providecommand\Matlab{\textsc{Matlab}}

%%%%%%%%%%%%%%%%%%%%%%%% End Helper Commands %%%%%%%%%%%%%%%%%%%%%%%%%%%

%%%%%%%%%%%%%%%%%%%%%%%%% Begin CV Document %%%%%%%%%%%%%%%%%%%%%%%%%%%%

\begin{document}

\makeheading{}{Taehwan Kim}

% NOTE: Mind where the & separators and \\ breaks are in the following
%       table.
%
% ALSO: \rcollength is the width of the right column of the table
%       (adjust it to your liking; default is 1.85in).
%
\vspace{.1in}
\newlength{\rcollength}
\setlength{\rcollength}{1.55in}

\hspace*{-\marginparsep minus \marginparwidth}
\begin{minipage}[t]{\textwidth+\marginparwidth+\marginparsep}%
\centering{
\begin{tabular}[t]{@{}>{\raggedleft}m{0.5\textwidth-0.5\marginparwidth}!{\vrule width 0.25pt}>{\raggedright}m{0.5\textwidth-0.5\marginparwidth}@{}}
\multicolumn{2}{c}{Berkeley Wireless Research Center, 2108 Allston Way, Berkeley, CA 94720}   \\[0.05in]
     +1-510-295-3549 & \email{taehwan@berkeley.edu}
\end{tabular}

\vspace{.07in}
%\sectionline{black}{85}
%\myrule[line width = 0.4mm]{hooks}{hooks}
\rule{\columnwidth}{1.4pt}
}
\end{minipage}

\vspace{-.07in}
%\section{Objective}

%Insert text here if you want to
%\begin{innerlist}
%\item More information and auxiliary documents can be found at\\\url{http://www.tedpavlic.com/facjobsearch/}
%\end{innerlist}

\section{Research Interests}
Electronic-photonic integrated systems for communication/sensing \\
Electronic-photonic integration technology \& design methodology development\\
Analog/mixed-signal integrated circuit design
    
\section{Education}
%\begin{outerlist}
%
%\item[] Ph.D.,
%        \href{http://www.sph.umn.edu/biostatistics/}
%             {Biostatistics},
%             \emph{Expected:} Summer 2013
%        \begin{innerlist}
%        \item Thesis Topic: \emph{Spatiotemporal Gradient Modeling with Applications}
%        \item Advisors:
%              \href{http://www.biostat.umn.edu/~brad/}
%                   {Bradley P. Carlin, Ph.D} and
%              \href{http://www.biostat.umn.edu/~sudiptob/}
%                   {Sudipto Banerjee, Ph.D}
%        \end{innerlist}
%
%\item[] M.S.,
%        \href{http://www.sph.umn.edu/biostatistics/}
%             {Biostatistics},
%             Aug 2010
%        \begin{innerlist}
%        \item Topic: \emph{Assessing the Impact of the Density of Alcohol Establishments on Crime in Minneapolis Neighborhoods Using Univariate and Multivariate Conditionally Autoregressive Models}
%        \item Advisor:
%              \href{http://www.biostat.umn.edu/~brad/}
%                   {Bradley P. Carlin, Ph.D}
%        \end{innerlist}
%\end{outerlist}
%\vspace{.1in}

\href{http://www.snu.ac.kr}{\textbf{University of California, Berkeley}} \hfill \textit{Aug. 2014 to Present}
%(\textit{Anticipated})
\begin{outerlist}
\item[] Ph.D. Student in Electrical Engineering and Computer Science
\end{outerlist}
\href{http://www.snu.ac.kr}{\textbf{Seoul National University}}\hfill \textit{Mar. 2007 to Feb. 2014}
%(\textit{Anticipated})
\begin{outerlist}
\item[] B.S. in \href{http://ece.snu.ac.kr} Electrical and Computer Engineering
\item[] B.A. in \href{http://econ.snu.ac.kr} Economics (Double major)
\end{outerlist}


\section{Research Experience}
\textbf{Graduate Student Researcher}  \hfill \textit{Aug. 2014 to present}
\begin{outerlist}
\item[] Integrated Systems Group, University of California, Berkeley (Advisor: Vladimir Stojanovi{\'c})
	\begin{innerlist}
     \item Optical phased array based systems in electronic-photonic heterogeneous integration platform
     \begin{innerlist}
     \item[-] Developing single-chip solution for ultra high-resolution laser radar (LIDAR) and free-space optical communication links leveraging optical phased arrays
     \item[-] Tape-out in early 2016 (65nm 10LPe process)
     \end{innerlist}
     \item Model-predictive control based algorithm for equalization of high-speed links
     \begin{innerlist}
     \item[-] Transmitter-side algorithm based on digital channel models for modular, energy-efficient equalization of asymmetric high-speed interfaces
     \item[-] Built chips in 28nm FDSOI \& 45nm SOI process
     \item[-] Chip measurement in progress
     \end{innerlist}
     \end{innerlist}
     \end{outerlist}

\textbf{Undergraduate Researcher}  \hfill \textit{Jun. 2012 to Feb. 2014}
\begin{outerlist}
\item[] \href{http://mics.snu.ac.kr}{Mixed-Signal IC and System Group}, Seoul National University (Advisor: Jaeha Kim)
	\begin{innerlist}
     \item Formal verification of analog/mixed-signal circuits 
     \begin{innerlist}
     \item[-] Developed an algorithm to verify the correct start-up behavior of ring oscillators in presence of variability
     \item[-] Implemented GCHECK: a Python-based tool for detection of start-up failures of coupled ring oscillators (transferred to Samsung Electronics)
     \end{innerlist}
     \item Variability-aware circuit optimization
     \begin{innerlist}
     \item[-] Developing global optimizer for analog/mixed-signal circuits based on statistical metamodeling
     \end{innerlist}
     \end{innerlist}     
\end{outerlist}

\section{Publications} \label{Sec:PUB}
{\bf T. Kim}, D.-G. Song, S. Youn, J. Park, H. Park, and J. Kim, ``Verifying Start-Up Failures in Coupled Ring Oscillators in Presence of Variability Using Predictive Global Optimization," in \emph{Proc. IEEE/ACM International Conference on Computer-Aided Design (ICCAD)}, 2013. \\
\vspace{-.6\baselineskip}

J. Kim, J. Lee, D.-G. Song, {\bf T. Kim}, K.-H. Kim, S. Jung, and S. Youn, ``Discretization and Discrimination Methods for Design, Verification, and Testing of Analog/Mixed Signal Circuits," in \emph{Proc. Custom Integrated Circuits Conference (CICC)}, 2013.

% Add a little space to nudge next ``Conference Publications'' marginpar
% down to make room for tall ``Submitted Journal Publications''
% marginpar. If there are enough submitted journal publications, this
% space will not be needed (and should be removed).
%\vspace{0.1in}

\section{Honors \& Awards}
Kwanjeong Scholarship for Abroad Studies \hfill \textit{2014-2018}\\
National Scholarship for Science and Engineering, Korea Science Foundation \hfill \textit{2007-2013}\\

\section{Skills}
%\begin{comment}
Languages: C, C++, Python, Verilog\\
Tools: Custom IC (Virtuoso, ADS) and VLSI design tools (DC, ICC, RC, SOC-ENC), \Matlab\\
Operating Systems: OSX, Linux, Windows 
%\end{comment}

\end{document}

%%%%%%%%%%%%%%%%%%%%%%%%%% End CV Document %%%%%%%%%%%%%%%%%%%%%%%%%%%%%

%----------------------------------------------------------------------%
% The following is copyright and licensing information for
% redistribution of this LaTeX source code; it also includes a liability
% statement. If this source code is not being redistributed to others,
% it may be omitted. It has no effect on the function of the above code.
%----------------------------------------------------------------------%
% Copyright (c) 2007, 2008, 2009, 2010, 2011 by Theodore P. Pavlic
%
% Unless otherwise expressly stated, this work is licensed under the
% Creative Commons Attribution-Noncommercial 3.0 United States License. To
% view a copy of this license, visit
% http://creativecommons.org/licenses/by-nc/3.0/us/ or send a letter to
% Creative Commons, 171 Second Street, Suite 300, San Francisco,
% California, 94105, USA.
%
% THE SOFTWARE IS PROVIDED "AS IS", WITHOUT WARRANTY OF ANY KIND, EXPRESS
% OR IMPLIED, INCLUDING BUT NOT LIMITED TO THE WARRANTIES OF
% MERCHANTABILITY, FITNESS FOR A PARTICULAR PURPOSE AND NONINFRINGEMENT.
% IN NO EVENT SHALL THE AUTHORS OR COPYRIGHT HOLDERS BE LIABLE FOR ANY
% CLAIM, DAMAGES OR OTHER LIABILITY, WHETHER IN AN ACTION OF CONTRACT,
% TORT OR OTHERWISE, ARISING FROM, OUT OF OR IN CONNECTION WITH THE
% SOFTWARE OR THE USE OR OTHER DEALINGS IN THE SOFTWARE.
%----------------------------------------------------------------------%
